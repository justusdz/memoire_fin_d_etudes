\newpage
\chapter*{Conclusion générale et perspectives}
Sont cachés dans des e-mails, derrière des liens ou des bannières, dissimulés dans des fichiers et des programmes téléchargés le plus légalement du monde …, les malwares, un nom générique qui désigne n'importe quelle forme de code malveillant, qu'il s'agisse d'un virus, un cheval de Troie (trojan), un keylogger, un spyware, un rootkit, etc.\\

Parmi les outils de protection contre les malwares, on trouve les antivirus. Les antivirus sont des logiciels conçus pour identifier, neutraliser et éliminer des logiciels malveillants. Ces derniers peuvent se baser sur l'exploitation de vulnérabilités des logiciels, mais il peut également s'agir de logiciels modifiant ou supprimant des fichiers, que ce soit des documents de l'utilisateur stockés sur l'ordinateur infecté, ou des fichiers nécessaires au bon fonctionnement de l'ordinateur.\\

L'analyse de malwares nous permet de bien comprendre le fonctionnement des malwares et pour générer des signatures efficaces afin d'éliminer les souches de malwares .\\

C'est pour cela que dans notre projet on a développé un antivirus qui sera capable de détecter les souches de malwares par la technique de détection par signature, qui est la technique la plus utilisée actuellement par les antivirus, ainsi qu'un parseur de PE qui nous permet d'observer le contenu d'un fichier exécutable. Avec ce parseur,  nous pouvons visualiser et examiner les fichiers de format PE ainsi que leurs structures internes.\\

\section*{Perspectives}

Parmi les perspectives qui restent à explorer, nous pouvons citer :\\
\begin{list}{•}{}
\item Concevoir d'autres techniques de détection telle que la technique comportementale
\item Détection des malwares en temps réel
\item Contrôle Parental
\item Ajouter à l'antivirus une fonctionnalité faisant objet d'un pare-feu
\item Permettre le chiffrement de fichiers
\item Et enfin, faire analyser la conception et le code source du projet par des experts en audit de sécurité, afin de corriger les éventuelles failles de sécurité qui nous ont échappées. Puis, effectuer des tests de pénétration et valider la sûreté de l’application.

\end{list}

