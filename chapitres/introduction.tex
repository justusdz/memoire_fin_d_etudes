\newpage
\chapter*{Introduction générale}


Devant le grand nombre de malwares identifiés chaque jour, des entreprises se sont spécialisées dans leur identification et éradication. Ce sont les éditeurs d'antivirus.\\

Les cybercriminels emploient désormais des techniques d'obfuscation de plus en plus complexes afin de rendre plus difficile l'identification de la nature malveillante des souches de malwares ainsi que la génération d'une signature utile à leur détection.\\

Des techniques d'infection novatrices sont également employées, ce qui nécessite de plus grands efforts de la part des éditeurs d'antivirus pour les désinfections. Afin d'identifier et éviter toute infection, les antivirus sont obligés d'opérer de lourdes modifications au niveau des systèmes d'exploitation.\\

Par conséquent, le déploiement d'un antivirus, provenant principalement d'Europe de l'Est ou des États Unis, dans un milieu professionnel (gouvernement, entreprises, associations, ...) ou personnel nécessite une confiance totale en ces états et pose,par conséquent,un sérieux problème relatif à la souveraineté nationale.\\

Dans le but de développer des compétences nationales dans le domaine de la lutte contre les malwares, notre projet englobera les points suivants :\\
\begin{itemize}


\item Introduction générale sur les malawres et les antivirus
\item L'établissement d'un état de l'art sur les mécanismes utilisés par les antivirus pour la détection et éradication des malwares
\item Étude sur le format PE (Portable Executable)
\item Le développement d'une preuve de concept d'une solution antivirale
\item L'implémentation de la preuve de concept de la solution antivirale
\item L'étude manuelle de plusieurs souches de malwares en vue de l'établissement d'une méthodologie d'analyse.
\end{itemize}
